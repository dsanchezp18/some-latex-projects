\documentclass[t,9pt,xcolor=dvipsnames]{beamer}
\usepackage{setspace}
\setstretch{1.5}
\usepackage[utf8]{inputenc}
\usepackage[spanish,es-lcroman,es-nosectiondot]{babel}%quitar el punto después de secciones
%\usefonttheme{professionalfonts}
\usepackage{nicefrac}
\usetheme{Berlin}
\useoutertheme{infolines}
\useinnertheme{rounded}
\definecolor{darkergreen}{rgb}{0.12, 0.3, 0.17}
\definecolor{lapislazuli}{rgb}{0.15, 0.38, 0.61}
\definecolor{veige}{rgb}{0.12,,215}
\definecolor{black}{rgb}{0,0,0}
\definecolor{burlywood}{rgb}{0.87, 0.72, 0.53}
\setbeamercolor{background canvas}{bg=white}


\usepackage{tikz}
\usepackage{pgfplots}
\pgfplotsset{width= 0.8 \textwidth,
            height=6 cm,
            compat=newest,
                grid=major,
    grid style= dashed,
    every tick label/.append style={font=\scriptsize}
}



\usecolortheme[named=black]{structure}
	\definecolor{cambridgeblue}{rgb}{0.64, 0.76, 0.68}
	\definecolor{lavenderblue}{rgb}{0.8, 0.8, 1.0}

\usepackage[style=authoryear,backend=biber,citestyle=apa,maxnames=2]{biblatex}
\DefineBibliographyStrings{spanish}{andothers={et~al\adddot}}%et al en vez de y cols.
\addbibresource{biblio.bib}

\renewcommand*{\bibfont}{\tiny}

\usepackage{amssymb,amsmath,tabu}
\usepackage{graphicx}
\usepackage{booktabs}
\title[Proyecto Final: MEG Experimental Modificado]{\textsc{Diseño Experimental: Juego del Mínimo Esfuerzo}}

\usepackage{multirow}
\subtitle{Proyecto Final Economía Experimental NRC 1441}
\author[Alejandra Marchán \& Daniel Sánchez]{Alejandra Marchán \and Daniel Sánchez}
\institute[USFQ]{Universidad San Francisco de Quito}
\date{Diciembre 2020}

\begin{document}
\begin{frame}
\titlepage
\end{frame}
\begin{frame}{Contenidos}
\begin{minipage}[c]{0.5 \textwidth}
\vspace{0.5cm}
\tableofcontents
\end{minipage}
\begin{minipage}[c]{0.45\textwidth}
\centering 
\includegraphics[scale=0.25]{monjas.jpg}
\end{minipage}
\end{frame}

\section{Introducción}
\subsection{Base Teórica}
\begin{frame}{El trabajo en grupos}
    \begin{itemize}
        \item Todo tipo de transacción económica involucra organización grupal
        \item Relevancia en la investigación y la academia así como en el ámbito organizacional
        \item \textcite{Nalbantian.1997} establecen que en los últimos años se ha creído que los grupos de trabajo han sido la manera de aumentar productividad
        \item Sin embargo, no siempre esto es así: las reglas y circunstancias alrededor del grupo son importantes en determinar el desempeño grupal
        \item Diseñamos un experimento económico para investigar el efecto de diferentes circunstancias en el desempeño del trabajo grupal
        \item Juego del Mínimo Esfuerzo (MEG): introducido por \Textcite{vanHuyck.1990} [VBB]
        \end{itemize}
\end{frame}
\begin{frame}{Minimum effort/weak-link game o Stag hunt variation}
\begin{itemize}
        \item Desempeño grupal puede estar determinado por el esfuerzo mínimo de los integrantes del grupo
    \item Juego estático en donde se decide en un nivel de esfuerzo
    \item Pagos determinados por el mínimo esfuerzo con la siguiente función:
    \begin{block}{Función de pagos de \textcite{vanHuyck.1990}}
    $$\pi(e_i)=a \left[\min(e_1, e_2,..., e_n) \right]-b e_i, \quad a>b>0$$
    Para $n$ jugadores, donde $e_i \in \{ 1,2,...,7\}$
    \end{block}
    \item Existe un equilibrio de Nash para cada nivel de esfuerzo, pero el de esfuerzos más altos es Pareto-dominante
    \item La decisión racional es que todos los jugadores escojan el nivel más alto de esfuerzo, pero no pasa ya que $\max(e_i)$ es riesgoso
\end{itemize}
\end{frame}
\subsection{Regularidades Empíricas}
\begin{frame}{Resultados del MEG en experimentos}
\begin{itemize}
        \item VBB \parencite*{vanHuyck.1990} demuestran que jugadores no logran converger en el equilibrio más alto, más bien quedan atrapados en un ``mal'' equilibrio.
         \begin{figure}[H]
 \centering
     \begin{tikzpicture}
      \begin{axis}[
    xlabel={Etapas},
    ylabel={Frecuencia},
    legend pos=north west,
    legend columns=2,
    legend entries={$e_i=1$, $e_i=7$},
    ylabel style={font=\small},
    xlabel style={font=\small},
    legend style={font=\small},
    ytick style={font= \small},
    xtick style={font= \small},
    ytick={0,10,20,30,40,50,60},
      ]
            \addplot[red, mark=square] table {unos.txt};
            \addplot[blue, mark=diamond] table {sietes.txt};
            \end{axis}
     \end{tikzpicture}
       \caption{Frecuencia de esfuerzos en el experimento de VBB (1990), tratamiento $A$.}
 \end{figure}
\end{itemize}
   \end{frame}
   \begin{frame}{Resultados del MEG en experimentos}
   \begin{itemize}
   \item Modificar el tamaño de grupo afecta a la coordinación y por ende a los niveles de esfuerzo escogidos por los miembros del grupo
   \item El tipo de sujetos en la muestra afecta a los niveles de esfuerzo escogidos
   \begin{itemize}
       \item \textcite{Weidenholzer.12282012}: niños de 7-10 años apuntan a tener mayores niveles de esfuerzo pero no logran coordinar
       \item \textcite{Engelmann.2010}: el \% de jugadores de Dinamarca positivamente relacionado con los niveles de esfuerzo en los grupos del MEG
         \end{itemize}
             \item \textcite{Gachter.2015}: la presencia de relaciones pre-existentes en el juego aumenta niveles de esfuerzo
       \item \textcite{Holt.2007}: un costo del esfuerzo $b$ más alto reduce niveles de esfuerzo
       \item \textcite{Chaudhuri.2009} la posibilidad de jugadores de comunicarse entre sí solo es significativa cuando los consejos son información común
   \end{itemize}
     \end{frame}
     \section{Hipótesis Experimentales}
     \begin{frame}{Hipótesis Experimental}
     \begin{itemize}
            \item La facilidad de coordinación es clave para determinar los niveles de esfuerzo
     \item En la pandemia, los grupos de trabajo pueden tener serios problemas de coordinación
     \item La coordinación en métodos virtuales se suele ver como inferior a la presencial
     \item Comparar resultados experimentales contra controles (coordinación presencial) así como contra otros resultados experimentales. 
     \begin{block}{Pruebas de hipótesis: Coordinación virtual vs. presencial}
     \begin{align*}
          H_0&: \mu_{cv}=2,79 &&   H_0: \mu_{cv}= \mu_{cp} \\
    H_1&: \mu_{cv}< 2.79 &&  H_1: \mu_{cv}< \mu_{cp}
     \end{align*}
              \end{block}
     
                   \end{itemize}
     \end{frame}
\section{Diseño Experimental}
 \subsection{Pagos y formación de grupos}
 \begin{frame}{Pagos del experimento}
 \begin{itemize}
      \item Imitamos en lo más posible al experimento de VBB (1990)
      \item En $\pi(e_i)$, definimos $a=20$, $b=10$ y agregamos una constante $c=70$, así como la columna con $e_i=0$:
      \begin{table}[H]
      \centering
      \small
\begin{tabular}{l|llllllll}
\toprule
\multirow{2}{*}{Tu número} & \multicolumn{8}{c}{Mínimo del número de tus compañeros} \\
                           & 0     & 1    & 2    & 3     & 4     & 5     & 6     & 7     \\
                           \midrule
7                          & 0     & 20   & 40   & 60    & 80    & 100   & 120   & 140   \\
6                          & 10    & 30   & 50   & 70    & 90    & 110   & 130   & 130   \\
5                          & 20    & 40   & 60   & 80    & 100   & 120   & 120   & 120   \\
4                          & 30    & 50   & 70   & 90    & 110   & 110   & 110   & 110   \\
3                          & 40    & 60   & 80   & 100   & 100   & 100   & 100   & 100   \\
2                          & 50    & 70   & 90   & 90    & 90    & 90    & 90    & 90    \\
1                          & 60    & 80   & 80   & 80    & 80    & 80    & 80    & 80    \\ \bottomrule
\end{tabular}
\end{table}
 \end{itemize}
 \end{frame}
\begin{frame}{Formación de los grupos}
\begin{minipage}[c]{0.7\textwidth}
\begin{itemize}
  \item Grupos de $P=16$ sujetos, se forman aleatoriamente después de cada ronda; 15  regulares y un ``impostor'' con pagos diferentes:
    \end{itemize}
    \begin{table}[H]
    \centering
    \small
    
\begin{tabular}{l|lllllll}
\toprule
\multicolumn{1}{c}{\multirow{2}{*}{Tu número}} & \multicolumn{7}{c}{Mínimo de los números   de tus compañeros} \\\multicolumn{1}{c}{}                             & 1      & 2      & 3      & 4       & 5      & 6      & 7      \\ \midrule
7                                                & 0      & 0      & 0      & 0       & 0      & 0      & 0      \\
6                                                & 0      & 0      & 0      & 0       & 0      & 0      & 0      \\
5                                                & 0      & 0      & 0      & 0       & 0      & 0      & 0      \\
4                                                & 0      & 0      & 0      & 0       & 0      & 0      & 0      \\
3                                                & 0      & 0      & 0      & 0       & 0      & 0      & 0      \\
2                                                & 0      & 0      & 0      & 0       & 0      & 0      & 0      \\
1                                                & 0      & 0      & 0      & 0       & 0      & 0      & 0      \\
0                                                & 70     & 80     & 90     & 100     & 120    & 140    & 150   \\ \bottomrule
\end{tabular}
\end{table}
    \end{minipage}
    \hfill
    \begin{minipage}[c]{0.2 \textwidth}
\includegraphics[scale=0.2]{impostr.png}
\end{minipage}

\end{frame} 
\subsection{Tratamientos}
\begin{frame}{Tratamientos}
\begin{block}{Tratamientos para diseño experimental MEG modificado}
    \begin{table}[H]
\centering 
\begin{tabular}{l|l|lllll}
\toprule
Características             & $T_1$      & $T_2$      & $T_3$     & $T_4$      & $T_5$     & $T_6$      \\
\midrule 
Tamaño de Grupos            & \multicolumn{6}{c}{15}
\\ 
Coordinación                & Ninguna & Ninguna & Física & Virtual & Física & Virtual \\
Tiempo de Coord.      & 0       & 0       & 1      & 1       & 2      & 2       \\
Etapas                      & \multicolumn{6}{c}{10}                                  \\
Duración Aprox. & 1       & 1       & 2      & 2       & 3      & 3       \\
Impostores                  & 0       & \multicolumn{5}{c}{1}  \\
Emparejamiento & \multicolumn{6}{c}{Aleatorio} \\
\bottomrule
\end{tabular}
\end{table}
\end{block}
\end{frame}
\subsection{Muestreo}
\begin{frame}{Tamaño de la muestra para intervalos de confianza}
    \begin{itemize}
        \item Para establecer intervalos de confianza de los esfuerzos promedio para cada tratamiento utilizamos \textcite{Anderson.2011}:
          \end{itemize}
            \begin{block}{Tamaño de muestra para intervalos de confianza para una media poblacional} 
                $$n=\dfrac{(z_{\nicefrac{\alpha}{2}})^2\cdot \sigma^2}{E^2}$$
                Donde $n$ es el número de \textit{observaciones} y con $E=\nicefrac{1}{2}\cdot w$ de un intervalo al $(1-\alpha)\%$ de confianza
        \end{block}
\begin{itemize}
    \item Trataremos de replicar la exactitud de VBB (1990), donde $E\approx 0,15 $ y $\sigma_V\approx1.85$, o realizar un experimento piloto
    \item Con esta fórmula, $n=428$; se requeriría $\nicefrac{428}{10}\approx 43$ sujetos
    \item Para completar 3 grupos de 16 se requeriría 45 sujetos regulares y 3 impostores, en total 48 sujetos a ser reclutados
\end{itemize}
\end{frame}
\begin{frame}{Tamaño de la muestra para la prueba de hipótesis}
    \begin{itemize}
        \item Considerando tanto la confianza de una prueba de hipótesis  $1-\alpha$ como el poder $1-\beta$, se utiliza la formúla simplificada de \textcite{Jacquemet.2018} que proporcionan \textcite{Anderson.2011}
        \end{itemize}
        \begin{block}{Tamaño de muestra para pruebas de hipótesis  de \textcite{Anderson.2011}}
            $$n'=\dfrac{(z_{\alpha}+z_{\beta})^2 \cdot  \sigma^2}{(\mu_0-\mu_a)^2}$$
                    \end{block}
                    \begin{itemize}
                        \item Se define $\alpha=\beta=0,05$, $\sigma=\sigma_V$, $\mu_0=2,79$
                        \item La media comparable para errores tipo II la definimos como $\mu_a=2.5$
                        \item $n'=442$, lo que nos lleva a los mismos requerimientos anteriores
                       \end{itemize}
                       \end{frame}
\begin{frame}{Características de la muestra}
\begin{itemize}
    \item El experimento de VBB reclutó solamente estudiantes de la carrera de economía, sin embargo podríamos caer en problemas de validez externa
    \item \textcite{Jacquemet.2018} sostiene que los estudiantes de economía se comportan de manera diferente al resto de la población
    \item Se debería reclutar a estudiantes de la USFQ de todo tipo de carreras mediante diferentes medios, dando un pago inicial de participación
    \item Debería representarse para cada colegio académico las proporciones que se encuentran en la población de todos los estudiantes de la USFQ
\end{itemize}
    \end{frame}       
\section{Otras consideraciones}
\begin{frame}{Requerimientos Adicionales}
\begin{itemize}

\item Una clase/laboratorio por cada grupo de 16 sujetos que participan en el experimento
\item 3 personas para poder supervisar el proceso experimental
\item En caso de no poder contar con computadores, se necesitaría solicitar a sujetos que utilicen sus dispositivos para llevar a cabo el tratamiento $T_4$ y $T_6$
\item Nos basamos en las instrucciones de \textcite{Holt.2007} para el MEG, sin embargo neutralizamos el lenguaje como lo hace \textcite{Weidenholzer.12282012}, reemplazando ``pagos'' por ``puntos'' y ``esfuerzos'' por ``números''. 
\end{itemize}
\end{frame}
\section{Literatura citada}
\begin{frame}{Literatura Citada}
\printbibliography
    
\end{frame}
\end{document}
